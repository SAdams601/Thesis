% Documentation:
%
% The UoK class file extends the standard report style to follow the Registry
% guidelines for laying out a thesis. It sets the margins, interline spacing,
% the page, figure and table numbering style, and disallows page breaks at
% hyphens. The class file consists of setting one and an half line spacing text
% with a 4cm left margin, at least a 2.5cm right margin, approximately 2cm top
% and bottom margin, on A4 paper.
% 
% The class the following options, in addition to those of the standard report
% class.
%     mini - Toggles the thesis in to mini-thesis mode. This adds "mini" to the
%            title and appends a nocite(*) at the end for an automatic output of
%            your complete bibliography.
%     draftmark - Puts a DRAFT' watermark on every page of the document along
%                 with the draft statement on the title page. Additionaly, it
%                 is used as a switch for the UoKExtentions package.
%     draft - Puts the entire document into draft mode. Applies all the effect
%             of draftmark above, but also propergates to other packages used.
%     copyright - Adds a copyright page between the title page and the preface.
%     nofig - Disables output of the list of figures in the preface.
%     notab - Disables output of the list of tables in the preface.
% All options passed to UoKthesis will be passed along to included packages:
%    natbib, draftwatermark, setspace, hyperref, lmodern
%
% The cover page and optional copyright page are implicitly added before the
% start of the preface section. Use the following commands to populate the 
% cover page/copyright page information:
%     \title{thesis title}
%     \author{author's name} 
%     \degree{Master of Science, Doctor of Philosophy, etc.} 
%     \subject{author's department}
%          - Computer Science if omitted 
%     \submitdate{month year in which submitted}
%          - dated by LaTeX if omitted 
%     \copyrightyear{year degree conferred (next year if submitted in Dec.)}
%          - assumes current year (or next year, in December) if omitted 
% 
% The preface environment allows for the use of sections that precede the main
% document; such as Abstract and  Acknowlegements. These sections should be
% defined using \section{Preface Section Title}. The contents page (and list of
% figures and tables if in use) will be automatically inserted at the end of the
% preface environment.
%
% The thesis style invokes the setspace package to set the commands:
%     \doublespace
%     \onehalfspace
%     \singlespace
% for spacing. By default one and an half spacing is used which resembles the
% UKC Typewriter requirement. Singlespace can be used for letterpress
% appearance. If you want to use true double space, for some reason, place the
% \doublespace command where you want to start using double spacing. Just call
% the appropriate spacing command at where you want to use them.
% 
% In the figure and table environments, single spacing is used. If you want to
% use any other size rather than one and an half spacing, then do:
% 	\renewcommand{\baselinestretch}{1.6} (or whatever you want instead of 1.6)
% This command won't take effect unless it comes before the \begin{document} or
% is triggered by a font change (after something like \small \normalsize).
%
% The example below shows the 12pt thesis style being used. This seems to give
% acceptable looking results, but it may be omitted to get 10pt. Alternatively,
% the 11pt option can be used.
%
% This version differs from old_ukcthesis.sty in the following ways:
% 1. Removed the doublespace package (now uses setspace).
% 2. Merged the phantom section for correct PDF links into the bibliography
%    generating function. 
% 3. Added thesis type options (mini, draft).
% 4. Kent Harvard is used for referencing and citation, this is supported by the
%    natbib package.
% 5. PsFig macro removed.
% 6. Now comes as two files, UoKthesis.cls, which defines purely stylistic layout,
%    and UoKextentions.sty, that provideds some additional functionality.

\documentclass[12pt,draft]{UoKthesis}

% Note: The UoKextentions package includes the xcolor package with the [usenames]
% options. If you need to add further options, these can be given to UoKextentions
% to be propogated through.
\usepackage{UoKextentions}

%\usepackage{llncsdoc}
%\usepackage{verbatim}
\usepackage{url}
\usepackage{color}
\usepackage{amsmath}
\usepackage{relsize}
\usepackage[final]{listings}
\usepackage[T1]{fontenc}
\usepackage[math]{iwona}
\newcommand{\comment}[1]{{\bf \tt  {#1}}}
\lstset{
  frame=none,
  xleftmargin=2pt,
  stepnumber=1,
  numbers=left,
  numbersep=5pt,
  numberstyle=\ttfamily\tiny\color[gray]{0.3},
  belowcaptionskip=\bigskipamount,
  captionpos=b,
  escapeinside={*'}{'*},
  language=haskell,
  tabsize=2,
  emphstyle={\bf},
  commentstyle=\it,
  stringstyle=\mdseries\rmfamily,
  showspaces=false,
  keywordstyle=\bfseries\rmfamily,
  columns=flexible,
  basicstyle=\small\sffamily,
  showstringspaces=false,
  morecomment=[l]\%,
}

% Kent Harvard Bibliography Style. WIP
\bibliographystyle{kentHarvard}

% Provides nice linking in PDFs
\usepackage{hyperref}

% Only needed if you want to produce an index. Example is shown at the bottom of this document.
\usepackage{makeidx}

% Useful packages
% \usepackage{epstopdf} % Converts EPS files to PDF using ghostscript
% \usepackage{fnbreak}  % Warns you if you have split footnotes
% \usepackage{mathpazo} % Type­set­ math­e­mat­ics in the Palatino fam­ily of text fonts
% \usepackage{paralist} % Enumerate and itemize within paragraphs
% \usepackage{amsmath}  % AMS mathematical facilities
% \usepackage{rotating} % Rotating facilities for floats


\setcounter{secnumdepth}{3} % add more section types

%%%%% macros
\def\fixme#1{\fbox{\textbf{\textsc{Fixme}}\quad#1}}
\def\fixpic#1{\fbox{\textbf{\textsc{Picture}}\quad#1}}
\def\defnx#1#2{\emph{#1}\index{#2}}
\def\defn#1{\defnx{#1}{#1}}
\def\floatpic#1#2{%
\begin{wrapfigure}{r}{\dimexpr #1 / 2 \relax}
\includegraphics[width=\dimexpr #1 / 2 \relax]{#2}
\end{wrapfigure}}
\def\inlinepic#1#2{%
\begin{center}
\includegraphics[width=\dimexpr #1 / 2 \relax]{#2}
\end{center}}

%%%%% augment hyphenation
\hyphenation{wide-spread}

%%%%% document start
\begin{document}

\title{Type-Changing Refactorings in Haskell}
\author{Stephen Adams}
\subject{Computer Science}
\degree{PhD}

\begin{preface}
\section{Abstract}
This mini-thesis tells you all you need to know about...
\section{Acknowledgements}
I would like to thank...
\end{preface}

\chapter{Introduction}
This is probably going to be written last.

\chapter{Refactoring Haskell in HaRe}
History of HaRe and the update to the GHC. Maybe some architectural details.

\chapter{Data refactoring in a functional context}
Introduce some of the ideas of functional data refactorings. Talk about simple refactoring examples like Hughes lists and Maybe generalising to MonadPlus.

\chapter{Generalising Monads to Applicative} 
In their 2008 functional pearl ``Applicative programming with effects" Conor McBride and Ross Paterson introduced a new typeclass that they called Idioms but are also known as Applicative Functors~\citep{mcbrideIdioms}. Idioms provide a way to run effectful computations and collect them in some way. They are more expressive than functors but more general than Monads, further work was done in~\citep{arrowsAndIdioms} to prove that Idioms are also less powerful than Arrows.

Applicative functors were implemented in the GHC as the typeclass \texttt{Applicative}. An interesting part of the history of the GHC is that despite McBride and Paterson proving in their original functional pearl that all monads are also applicative functors, however, the GHC did not actually require instances of monad to also be instances of Applicative until GHC's 7.10.1 release~\citep{ghc7.10Release}. Now that every monad must also be an applicative functor there now exists a large amount of code which could be rewritten using the applicative operators rather than the monadic ones. 

This chapter will discuss the design and implementation of a refactoring which will automatically refactor code written in a monadic style to use the applicative operators instead. Section~\ref{sec:appOverview} is a brief overview of the \texttt{Applicative} typeclass's operators, section~\ref{sec:appProgStyle} will discuss the applicative programming style and, in general, how programs are constructed using the applicative operators, next, section~\ref{sec:appApps} will cover some common applications of this refactoring, section~\ref{sec:appRefact} will specify the refactoring itself, section~\ref{sec:appPrecons} covers the preconditions of the refactoring, finally section~\ref{sec:appVariations} outlines other refactorings that may be used in conjunction with the generalising monads to applicative refactoring and some possible variations of this refactoring. 
\section{The Applicative Typeclass}
\label{sec:appOverview}

\todo{Write more of an introduction here about what exactly the applicative context is before going into the applicative operators.}

To implement the applicative typeclass two functions must be defined, \texttt{pure} and \texttt{(<*>)}. The types of these two functions are shown in listing~\ref{appTypes} where \texttt{f} is the applicative functor. 

\begin{lstlisting}[frame=tblr,label=appTypes,caption={Types of Applicative's minimal complete definition}]
pure :: a -> f a
(<*>) :: f (a -> b) -> f a -> f b
\end{lstlisting}

The \texttt{pure} function is the equivalent of monad's \texttt{return}, it simply lifts a value into the applicative context. The other function \texttt{(<*>)} (which is typically pronounced ``applied over" or just ``apply"). Apply take in two arguments, both of which are applicative values. The first argument is function within an applicative context from types a to b, and the second argument is of type a. Apply returns a value of type b inside of the same functional context. Apply ``extracts" the function from the first argument and the value from the second argument and applies it to the function all within whatever the applicative context is.

\subsection{Other useful functions}

Though \texttt{pure} and apply are the only two functions that are required to be defined to declare an instance of applicative there are several other useful functions that can either be derived from these two functions or come from other typeclasses which will be briefly covered here. First there are two variations on apply.

\begin{lstlisting}[frame=tblr]
(*>) :: f a -> f b -> f b
(<*) :: f a -> f b -> f a
\end{lstlisting}

These functions sequence actions and still perform the contextual effects of both of their arguments but discard the value of the first and second argument respectively. These functions are used when some operation affects the applicative context but their returned value will not affect the final result of the applicative expression. For example when writing parsers it is common to have to consume some characters from the input without those characters affecting the final result of the parser.

Another operator that is very useful for programming in the applicative style (discussed further in section~\ref{sec:appProgStyle}) is the infix synonym of \texttt{fmap}.

\begin{lstlisting}[frame=tblr]
(<$>) :: Functor f => (a -> b) -> f a -> f b
\end{lstlisting}

A consequence of the applicative laws is that every applicative's functor instance will satisfy the following~\citep{control.applicative}: 

\begin{lstlisting}[frame=tblr]
f <$> x = pure f <*> x
\end{lstlisting}

The next section will cover how these functions can be used in an applicative style of programming. 

\section{The Applicative Programming Style}
\label{sec:appProgStyle}

In~\cite{mcbrideIdioms} McBride and Paterson prove that any expression built from the applicative combinators can take the following canonical form:

\begin{lstlisting}[frame=tblr]
pure f <*> is_1 <*> ... <*> is_n
\end{lstlisting}

\todo{Include my variation on the canonical form which composes f with the infix fmap operator instead?}

Where some of the \texttt{is}'s have the form \texttt{pure s} for a pure function \texttt{s}. This is the form that most programs will take when they are refactored from a monadic style.

 Context-free parsing is a good use case of the applicative type and many examples in this chapter are taken from parsers defined using the parsec library~\citep{parsec}. The first example of the applicative programming style is a function that parses money amounts of the form \texttt{<currency symbol><whole currency amount>.<decimal amount>} e.g. ``\$4.59" or ``\textsterling64.56".
 
 \begin{lstlisting}[frame=tblr]
 data Currency = Dollar
                          | Pound
                          | Euro
              
data Money = M Currency Integer Integer

parseMoney :: CharParser () Money
parseMoney = M <$> parseCurrency <*> readWhole <*> readDecimal
 \end{lstlisting}
 
The \texttt{parseMoney} function is in the canonical form as defined by~\cite{mcbrideIdioms}. The pure function \texttt{M} is lifted into the \texttt{CharParser} context and its three arguments are provided by three smaller parsers that handle the currency symbol, the whole amount, and the decimal amount separately. 

The only difference between \texttt{readWhole} and \texttt{readDecimal} is that \texttt{readDecimal} has to consume the decimal point before reading the number. Instead of duplicating that number code let's perform a small refactoring to lift the parsing of the decimal into the \texttt{parseMoney} function which will allow us to reuse the \texttt{readWhole} function.

 \begin{lstlisting}[frame=tblr]
parseMoney :: CharParser () Money
parseMoney = M <$> parseCurrency <*> readWhole <* char '.' <*> readWhole
 \end{lstlisting}
 
 Here we can see that the result of parsing the decimal point is discarded because of the use of \texttt{<*} rather than the full apply. All of the variations of apply are left associative so the following definition of \texttt{parseMoney} causes a type error.
 
  \begin{lstlisting}[frame=tblr]
parseMoney :: CharParser () Money
parseMoney = M <$> parseCurrency <*> readWhole <*> char '.' *> readWhole
 \end{lstlisting}
 
This error can be corrected by wrapping \texttt{char '.' *> readWhole} in parenthesis.  
 
 

%%%%%%%%%%%%%%%%%%%%%%%%%%%%%%%%
 \iffalse
 The first example of the applicative programming style is a simple function that parses strings surrounded by double quotes.
 
\begin{lstlisting}[frame=tblr]
parseStr :: CharParser () String 
parseStr = char '"' *> (many1 (noneOf "\"")) <* char '"'
\end{lstlisting}

The calls to the \texttt{char} combinator consumes a single character of input succeeds if that character matches the parsers only argument, and then returns the matched character. In this case we don't care about the double quotes just the characters between them so the result of the parser is ignored (hence the use of the $*>$ and $<*$ operators). The parser who's result actually affects the final value of this function is the \texttt{many1 (noneOf "\textbackslash"")} combinator. This parses one or more characters excluding a double quote.

You may have noticed that this function is not in the canonical form but can be transformed to the canonical form by prepending the identity function like so:

\begin{lstlisting}[frame=tblr]
parseStr :: CharParser () String 
parseStr = pure id <*> char '"' *> (many1 (noneOf "\"")) <* char '"'
\end{lstlisting}

\fi

\section{Applications of the Refactoring}
\label{sec:appApps}
\todo{Maybe move this to earlier in the chapter? Since the previous section uses parsers as examples which is a major use case this section may end up being fairly redundant. Perhaps this should just be a subsection before section~\ref{sec:appOverview}}

\section{Refactoring Monadic Programs to Applicative}
\label{sec:appRefact}
\section{Preconditions of the Refactoring (When is a Monad actually a Monad?)}
\todo{Main precondition is that no variable assigned in the do block can be used on the RHS before the return statement. There can also only be a single return statement, e.g. no branches in computation.}
\label{sec:appPrecons}
\section{Variations and Related Refactorings}
\label{sec:appVariations}

\subsection{Inline do blocks}
\todo{Some examples of monadic code are only monadic in small chunks so the whole function may be able to take on more of an applicative structure but with a small do block embedded in it. See example below}

\begin{lstlisting}[frame=tblr]
f = do
	x <- result1
	y <- result2
	z <- result3
	log z
	return (x,y)
\end{lstlisting}
\larger[5]
\[\Rightarrow\]
\normalsize
\begin{lstlisting}[frame=tblr]
f = (,) <*> result1 <*> result2 <* do{z <- result3; log z}
\end{lstlisting}

\subsection{Reordering of monadic statements}
\todo{The statements in an applicative chain need to be ordered in the way the pure constructor takes the arguments. For example if the function \texttt{f :: Int -> Char -> String} was the pure constructor in the following monadic statement:}

\begin{lstlisting}[frame=tblr]
g = do
	x <- getChar
	y <- getInt
	return $ f y x
\end{lstlisting}

\todo{The generalisation refactoring would produce \texttt{g = f <*> getChar <*> getInt} which throws a type error. Swapping lines two and three before attempting the generalisation would fix this.}

\chapter{Mysterious third chapter of contribution}
More research goes here.

\chapter{Related work}
Mention type and transform, Meng Wang's paper...

\chapter{Conclusion}
Well its done..

\bibliography{main}

% This index section is optional, use cleardoublepage and phantomsection to make the links work in your contents page. Uses makeidx package.
\cleardoublepage
\phantomsection
\label{index}
\printindex

\end{document}
